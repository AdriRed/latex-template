\documentclass[11pt,a4paper]{article}
\usepackage[catalan]{babel}

% Paquetes necesarios
% preamble.tex

% Codificación y tipografía
\usepackage[utf8]{inputenc}
\usepackage[T1]{fontenc}
\usepackage[catalan]{babel}
\usepackage{lmodern}
\usepackage{subcaption}
% Márgenes y geometría
\usepackage{geometry}
\geometry{left=2.5cm, right=2.5cm, top=2cm, bottom=3cm}
% Estilo de página
\usepackage{fancyhdr}
\setlength{\headheight}{15pt}
\pagestyle{fancy}
\fancyhf{}
\lhead{\textit{Electrònica Física}}
\rhead{\textit{Adrià Rojo}}
\cfoot{\thepage}

% Gráficos y tablas
\usepackage{graphicx}
\usepackage{float}
\usepackage{caption}

% Matemáticas y símbolos
\usepackage{amsmath}
\usepackage{amssymb}

% Hipervínculos
\usepackage{hyperref}
\usepackage[per-mode=fraction]{siunitx}
% Espaciado
\usepackage{parskip}  % Quita sangría de párrafos y añade espacio
\usepackage{wrapfig}
\usepackage{floatflt}
\usepackage[backend=biber,style=ieee, autocite=superscript]{biblatex}
\addbibresource{bibliography.bib}

% Fuente y estilo general
\renewcommand{\familydefault}{\rmdefault}

%Definició de la ela geminada per tal que accepti el punt volat del teclat
% \def·#1{%
%   \ifmmode
%     \cdot #1
%     %\csname normal@char\string"\endcsname l%
%   \else%
%     \def\argument{#1}%
%     \if\argument l%
%       \leftllkern=0pt\rightllkern=0pt\raiselldim=0pt%
%       \setbox0\hbox{l}\setbox1\hbox{l\/}\setbox2\hbox{.}%
%       \advance\raiselldim by \the\fontdimen5\the\font
%       \advance\raiselldim by -\ht2%
%       \leftllkern=-.25\wd0%
%       \advance\leftllkern by \wd1%
%       \advance\leftllkern by -\wd0%
%       \rightllkern=-.25\wd0%
%       \advance\rightllkern by -\wd1%
%       \advance\rightllkern by \wd0%
%       \allowhyphens\discretionary{-}{l}%
%       {\hbox{}\kern\leftllkern\raise\raiselldim\hbox{.}%
%         \kern\rightllkern\hbox{l}}\allowhyphens%
%     \else
%       \if\argument L%
%         \leftllkern=0pt\rightllkern=0pt\raiselldim=0pt%
%         \setbox0\hbox{L}\setbox1\hbox{L\/}\setbox2\hbox{.}%
%         \advance\raiselldim by .5\ht0%
%         \advance\raiselldim by -.5\ht2%
%         \leftllkern=-.125\wd0%
%         \advance\leftllkern by \wd1%
%         \advance\leftllkern by -\wd0%
%         \rightllkern=-\wd0%
%         \divide\rightllkern by 6%
%         \advance\rightllkern by -\wd1%
%         \advance\rightllkern by \wd0%
%         \allowhyphens\discretionary{-}{L}%
%         {\hbox{}\kern\leftllkern\raise\raiselldim\hbox{.}%
%            \kern\rightllkern\hbox{L}}\allowhyphens%
%       \else
%         #1
%       \fi
%     \fi
%   \fi
%   }
% Título
\title{\textbf{CMOS: Història, fonaments i funcionament i aplicacions}}
\author{Adrià Rojo}
\date{\today}

\begin{document}

\maketitle
\thispagestyle{empty}
% Resumen
\begin{abstract}
    BLABLA
\end{abstract}

\section{Introducció}
% Introducir brevemente qué es CMOS, su importancia en la electrónica digital y qué estructura tendrá el trabajo.

Un semiconductor-òxid-metall complementari (\textit{Complementary-Metal-Oxide Semiconductor}) és un tipus de tecnologia que combina un parell simètric de N-MOS i P-MOS conectats de forma complementària per poder fer operacions lògiques de forma eficient \autocite{wiki:CMOS}. 

Aquesta tecnologia és la base d'electrònica moderna, sent el bloc principal de la porta lògica \texttt{NAND}, que és la base de tots els altres circuits existents i pot arribar formar circuits molt complexos com processadors i memòries RAM.

Els dispositius dissenyats amb el procés CMOS en ment també poden ser utilitzats com a sensor d'imatge, gràcies a la seva eficiència i rapidesa, formant part d'una àmplia varietat d'eines electròniques, des de telèfons mòbils a drons i càmeres fotogràfiques utilitzades en medicina o astronomia. 

Actualment s'estima que el nombre de transistors MOSFET manufacturats (inclou els CMOS) superen els $10^{22}$ dispositius\autocite{wiki:Transistor_count} al 2018, dada que està basada en la Llei de Moore.

A continuació, parlarem sobre la història de la tecnologia CMOS, els principis físics, el seu funcionament intern i les seves aplicacions actuals i futures.

\begin{figure}[h]
    \centering
    \includegraphics[width=0.6\linewidth]{images/moore.png}
    \caption{Gràfic de quantitat de transistors MOS per any. Wikipedia.}
    \label{<label>}
\end{figure}

\section{Història del CMOS}
% Destacar los hitos importantes desde 1963, comparaciones con tecnologías anteriores (NMOS, BJT).
% Mencionar su adopción masiva en los años 80-90 y cómo ha evolucionado.

\subsection{BJT i MOSFET}

Al final dels anys 1940, el transistor BJT va ser inventat per John Bardeen i Walter Brattain sota la direcció de William Shockley als laboratoris Bell Telephone. Gràcies a certes millores introduïdes de cara a l'eficiència, tant en el seu ús com a la seva producció, van ser introduïts al públic general als principis dels anys 1950 i van ser la tecnologia predominant al mercat durant trenta anys.

Prèviament, el transistor d'efecte de camp semiconductor-òxid-metall, o MOSFET, ja s'havia inventat i patentat a Europa. Aquest va ser objecte d'estudi per part de l'equip dels laboratoris Bell, però sense èxit, a causa dels problemes dels estats superficials \footnote{Els estats superficials són els estats electrònics que es formen a la superfície dels materials, deguts a la interrupció de la malla atòmica del material que acaba en la superfície. Aquest efecte es pot interrompre amb la \textit{passivització} de la superfície, que redueix l'efecte i estabilitza els estats electrònics al límit del material.} no va ser possible la seva adopció a gran escala.

Després de avenços amb el transistor (òxid de Silici com a passivitzador, invenció de la tecnología de fabricació planar), finalment Mohamed Atalla i Dawon Kahng van introduir el primer transistor MOS de silici funcional als laboratoris Bell l'any 1960. 

\begin{wrapfigure}{r}{0.3\paperwidth}
    \centering
    \includegraphics[width=\linewidth]{images/patent kahang.jpg}
    \caption{Patent estatunidenca del MOSFET de Dawon Kahng.}
\end{wrapfigure}

Tot i això, els transistors MOSFET no estaven a l'altura dels BJT de l'època i es vèien com inferiors. La principal diferència entre els MOSFET i BJT és el consum elèctric i el seu ús tipic: 
\begin{itemize}
    \item Els BJT son controlats directament amb corrent (Base) i aquesta controla el corrent entre el co$\cdot$ector i la font de manera lineal, fent-los ideals per a circuits analògics.
    \item Els MOSFET només consumeixen electricitat quan volen canviar l'estat, fet que els fa idonis pels circuits digitals.
\end{itemize}



Lorem ipsum dolor sit amet, consectetur adipiscing elit. Cras nulla lectus, varius sed consequat in, blandit vel ligula. Quisque in ultrices urna. Sed dictum et nisi a rutrum. Integer ullamcorper facilisis odio non luctus. Sed euismod ipsum massa, at tincidunt purus molestie vel. Aenean quis tempus ligula. Quisque fringilla nisi vitae mollis facilisis. Vivamus finibus viverra aliquam. Phasellus semper consectetur efficitur. Etiam faucibus eleifend est, semper lacinia justo laoreet nec. Aenean tincidunt lacinia fermentum.

Proin sed neque in diam pellentesque pellentesque. Integer vestibulum risus vitae rutrum tincidunt. Sed ornare quam vitae erat feugiat finibus. Donec at pharetra turpis. Nulla convallis sapien in elit gravida vehicula. Maecenas nisi erat, maximus sed ullamcorper vitae, pellentesque id magna. Ut nec feugiat sapien. Donec mattis finibus orci eu sodales. Nam scelerisque suscipit ipsum, sit amet condimentum elit pretium sed.

Sed dictum, erat quis elementum ornare, sem nisi facilisis dui, a mattis dui nulla a ante. Nullam vitae fringilla velit. Quisque molestie arcu eu risus rhoncus, nec imperdiet mauris commodo. Aenean mollis, ex in ullamcorper consequat, tellus lectus semper dui, sit amet gravida lorem urna at quam. Phasellus lobortis leo a nunc aliquam, in gravida mi molestie. Proin at auctor turpis. Nam dictum ex felis, eu pharetra erat suscipit at. Vestibulum in varius dui. Fusce vel neque ac dui aliquet commodo ut non mi. Aliquam id luctus dui, et porta turpis. Cras ullamcorper neque vitae consequat lacinia. Sed commodo dapibus dolor nec tempor. Praesent porta aliquam nulla, vel hendrerit velit congue sit amet. Duis et lacus dui. Fusce bibendum elit nec sem pharetra, sit amet cursus orci convallis. Nullam imperdiet est vel est placerat ultricies.

Praesent lacus dui, egestas at magna a, vestibulum placerat urna. Duis commodo elit tortor, sit amet convallis ex elementum ut. Quisque egestas convallis convallis. Phasellus ac odio eu ex posuere porta a eu nibh. Aliquam bibendum nulla nec lacinia euismod. Integer eu justo eu nunc eleifend iaculis et at lorem. Curabitur luctus lacus est, non euismod nunc condimentum vitae. Morbi ante sapien, rutrum eget sollicitudin ut, congue ut neque. In hac habitasse platea dictumst. Vestibulum arcu sem, aliquam eu ex ut, lobortis dictum justo. Cras tempor blandit eros, vel mollis odio ultricies eget. In interdum risus non tempor accumsan. Aliquam quis imperdiet justo.

Cras arcu ex, feugiat sed convallis eu, porttitor sit amet orci. Curabitur arcu risus, interdum sed dolor non, elementum iaculis neque. Etiam et laoreet sapien. Praesent a ligula blandit, vulputate sem quis, ornare odio. Morbi malesuada at augue at fermentum. Donec a volutpat est. Cras turpis turpis, tristique porta eleifend a, feugiat efficitur velit. Nunc tincidunt lorem vel auctor vehicula. Proin convallis urna vel dui egestas mollis. Proin a tellus id leo commodo luctus vel non ipsum. Fusce sed consectetur dolor. 

\subsection{Naixement del CMOS, primers dispositius}

\subsection{Actualitat}

\section{Funcionament intern}

\subsection{Estructura d'un CMOS}
% Describir un inversor básico nMOS/pMOS en serie con explicación de entrada/salida.

\subsection{Bandes d'energia}
% Explicar las bandas de valencia/conducción en nMOS y pMOS, con comentario sobre inversión de canal.
% Comentar brevemente el modelo de portadores mayoritarios.

\subsection{Corrent de portadors}
% Diferenciar entre flujo de electrones (nMOS) y huecos (pMOS). 
% Mencionar el principio de conducción dependiente del voltaje de puerta.

% \subsection{Consumo de Potencia}
% % Diferenciar entre consumo estático y dinámico.
% % Comentar sobre la eficiencia del CMOS frente a otras tecnologías.

\section{Aplicacions}

\subsection{Portes lògiques}

\subsection{Sensors i cameres digitals}
% Cámaras digitales, sensores industriales, robótica.

\printbibliography

\end{document}
