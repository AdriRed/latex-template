\documentclass{article}
\usepackage{amsmath}
\usepackage{geometry}
\usepackage{longtable}
\geometry{margin=1.5cm}
\begin{document}

\section*{Leptones}

Todos los leptones tienen: $L = 1, B = 0, I = 0, I_3 = 0, J^P = \frac{1}{2}^+$

\begin{longtable}{|c|l|c|c|c|}
\hline
Símbolo & Nombre & Masa (MeV) & Carga $Q$ & Familia \\
\hline
$e^-$ & Electrón & 0.511 & $-1$ & $L_e$ \\
$\mu^-$ & Muón & 105.7 & $-1$ & $L_\mu$ \\
$\tau^-$ & Tau & 1776.9 & $-1$ & $L_\tau$ \\
$\nu_e$ & Neutrino electrónico & $<2.2 \times 10^{-6}$ & $0$ & $L_e$ \\
$\nu_\mu$ & Neutrino muónico & $<0.17$ & $0$ & $L_\mu$ \\
$\nu_\tau$ & Neutrino tauónico & $<18.2$ & $0$ & $L_\tau$ \\
\hline
\end{longtable}

\vspace{1em}
\section*{Mesones}

Todos los mesones tienen: $B = 0, L = 0$

\begin{longtable}{|c|l|c|c|c|c|c|c|c|}
\hline
Símbolo & Nombre & Masa (MeV) & Quarks & Carga $Q$ & Isospín $I$ & $I_3$ & $J^P$ & $S/C$ \\
\hline
$\pi^+$ & Pión & 139.6 & $u\bar{d}$ & $+1$ & $1$ & $+1$ & $0^-$ & $0$ \\
$\pi^0$ & Pión neutro & 135.0 & $(u\bar{u} - d\bar{d})/\sqrt{2}$ & $0$ & $1$ & $0$ & $0^-$ & $0$ \\
$K^+$ & Kaón & 493.7 & $u\bar{s}$ & $+1$ & $\frac{1}{2}$ & $+\frac{1}{2}$ & $0^-$ & $S=-1$ \\
$K^0$ & Kaón neutro & 497.6 & $d\bar{s}$ & $0$ & $\frac{1}{2}$ & $-\frac{1}{2}$ & $0^-$ & $S=-1$ \\
$\eta$ & Eta & 547.9 & mezcla $q\bar{q}$ & $0$ & $0$ & $0$ & $0^-$ & $0$ \\
$\rho^+$ & Rho & 775.3 & $u\bar{d}$ & $+1$ & $1$ & $+1$ & $1^-$ & $0$ \\
$K^{*0}$ & Kaón estrella & 896 & $d\bar{s}$ & $0$ & $\frac{1}{2}$ & $-\frac{1}{2}$ & $1^-$ & $S=-1$ \\
$\phi$ & Phi & 1019 & $s\bar{s}$ & $0$ & $0$ & $0$ & $1^-$ & $0$ \\
\hline
\end{longtable}

\vspace{1em}
\section*{Bariones}

Todos los bariones tienen: $ L =  0, B = 1$

\begin{longtable}{|c|l|c|c|c|c|c|c|c|c|}
\hline
Símbolo & Nombre & Masa (MeV) & Quarks & Carga $Q$ & Isospín $I$ & $I_3$ & $J^P$ & $S/C$ \\
\hline
$p$ & Protón & 938.3 & $uud$ & $+1$ & $\frac{1}{2}$ & $+\frac{1}{2}$ & $\frac{1}{2}^+$ & $0$ \\
$n$ & Neutrón & 939.6 & $udd$ & $0$ & $\frac{1}{2}$ & $-\frac{1}{2}$ & $\frac{1}{2}^+$ & $0$ \\
$\Lambda^0$ & Lambda & 1115.7 & $uds$ & $0$ & $0$ & $0$ & $\frac{1}{2}^+$ & $S=-1$ \\
$\Sigma^+$ & Sigma & 1189.4 & $uus$ & $+1$ & $1$ & $+1$ & $\frac{1}{2}^+$ & $S=-1$ \\
$\Xi^0$ & Xi neutro & 1314.9 & $uss$ & $0$ & $\frac{1}{2}$ & $+\frac{1}{2}$ & $\frac{1}{2}^+$ & $S=-2$ \\
$\Omega^-$ & Omega & 1672.4 & $sss$ & $-1$ & $0$ & $0$ & $\frac{3}{2}^+$ & $S=-3$ \\
$\Delta^{++}$ & Delta & 1232 & $uuu$ & $+2$ & $\frac{3}{2}$ & $+\frac{3}{2}$ & $\frac{3}{2}^+$ & $0$ \\
\hline
\end{longtable}

\vspace{1em}
\section*{Nota sobre partículas resonantes (*)}

Las partículas con asterisco como $K^*$, $\rho$, $\Delta$, $\Sigma^*$ son **estados excitados de corta vida**, conocidos como **resonancias hadrónicas**.  
Se desintegran rápidamente (por interacción fuerte) en sus correspondientes partículas fundamentales.

\end{document}
