\documentclass[10pt,twocolumn]{article}
\usepackage{amsmath}
\usepackage{geometry}
\usepackage{supertabular} % Alternativa a supertabular
\usepackage[compact]{titlesec}
\usepackage{booktabs}
\usepackage{amssymb}
\usepackage[table]{xcolor}
\usepackage{multirow}
\usepackage{graphicx}
\usepackage{enumitem} % Paquete recomendado para mejor control

\definecolor{weak}{RGB}{255,153,102}
\definecolor{em}{RGB}{102,178,255}
\definecolor{strong}{RGB}{153,255,153}
\definecolor{grav}{RGB}{255,204,255}

\geometry{margin=0.7cm}
\AtBeginEnvironment{supertabular}{\small}

\begin{document}


\subsection*{Reglas de Identificación}

\begin{supertabular}{|l|l|}
\hline
\rowcolor{gray!20} 
\textbf{Indicador} & \textbf{Interacción} \\
\hline
Cambio de sabor (quark) & \cellcolor{weak}Débil \\
Neutrinos involucrados & \cellcolor{weak}Débil \\
Decaimiento $\beta$ (e.g. $n \to p e^- \bar{\nu}_e$) & \cellcolor{weak}Débil \\
Emisión/absorción de fotones & \cellcolor{em}Electromagnética \\
Interacción con carga eléctrica & \cellcolor{em}Electromagnética \\
Producción de hadrones & \cellcolor{strong}Fuerte \\
Confinamiento de quarks & \cellcolor{strong}Fuerte \\
\hline
\end{supertabular}

\subsection*{Chequeo Rápido}

Para cualquier proceso, pregúntate:
\begin{enumerate}
\item ¿Hay cambio de sabor? $\Rightarrow$ Débil
\item ¿Hay hadrones en estado final/inicial? $\Rightarrow$ Fuerte
\item ¿Hay partículas cargadas sin cambio de sabor o fotones? $\Rightarrow$ EM
\end{enumerate}

\section*{Nomenclatura}

\subsection*{Símbolos y propiedades}
\begin{itemize}
\item $B$: Número bariónico
\item $L$: Número leptónico
\item $Q$: Carga eléctrica
\item $I$: Isospín
\item $I_3$: Tercera componente del isospín
\item $J^P$: Espín y paridad
\item $S$: Extrañeza
\item $C$: Conjugación de carga
\end{itemize}

\subsection*{Antipartículas}
\begin{itemize}
  \item Masa igual, carga opuesta
  \item $B$, $L$, $I_3$, $S$ cambian de signo
  \item Antileptones: $L = -1$
  \item Antibariones: $B = -1$
\end{itemize}

\subsection*{Conjugación de carga ($C$)}
\begin{itemize}
  \item Solo para partículas neutras
  \item Mesones neutros: $C = (-1)^{L+S}$
  \item $\gamma$, $g$: $C = -1$
  \item $W^\pm$ no son autoestados de $C$
\end{itemize}
\newpage
\section*{Leptones}
$B = 0$, $L = 1$, $I = 0$, $I_3 = 0$, $J^P = \frac{1}{2}^+$

\begin{supertabular}{|c|c|c|c|c|c|}
\hline
 & $m$ (MeV) & $Q$ & Familia & Quarks \\
\hline
$e^-$ & 0.511 & $-1$ & $L_e$ & - \\
$\mu^-$ & 105.7 & $-1$ & $L_\mu$ & - \\
$\tau^-$ & 1776.9 & $-1$ & $L_\tau$ & - \\
$\nu_e$ & $<2.2 \times 10^{-6}$ & $0$ & $L_e$ & - \\
$\nu_\mu$ & $<0.17$ & $0$ & $L_\mu$ & - \\
$\nu_\tau$ & $<18.2$ & $0$ & $L_\tau$ & - \\
\hline
\end{supertabular}

\section*{Mesones}
$B = 0$, $L = 0$

\begin{supertabular}{|c|c|c|c|c|c|c|c|}
\hline
 & $m$ (MeV) & $m^*$ (MeV) & $Q$ & $I$ & $J^P$ & $S$ & Quarks \\
\hline
$\pi^\pm$ & 139.6 & - & $\pm1$ & 1 & $0^-$ & - & $u\bar{d}$, $d\bar{u}$ \\
$\pi^0$ & 135.0 & - & $0$ & 1 & $0^-$ & - & $\frac{u\bar{u}-d\bar{d}}{\sqrt{2}}$ \\
$K^+$ & 493.7 & - & $+1$ & $\frac{1}{2}$ & $0^-$ & $+1$ & $u\bar{s}$ \\
$K^-$ & 493.7 & - & $-1$ & $\frac{1}{2}$ & $0^-$ & $-1$ & $s\bar{u}$ \\
$K^0$ & 497.6 & - & $0$ & $\frac{1}{2}$ & $0^-$ & $+1$ & $d\bar{s}$ \\
$\bar{K}^0$ & 497.6 & - & $0$ & $\frac{1}{2}$ & $0^-$ & $-1$ & $s\bar{d}$ \\
$\eta$ & 547.9 & - & $0$ & 0 & $0^-$ & - & $u\bar{u}$/$d\bar{d}$/$s\bar{s}$ \\
$\rho^+$ & 775.3 & - & $+1$ & 1 & $1^-$ & - & $u\bar{d}$ \\
$\phi$ & 1019 & - & $0$ & 0 & $1^-$ & - & $s\bar{s}$ \\
\hline
\end{supertabular}


\section*{Bariones}
$B = +1$, $L = 0$

\begin{supertabular}{|c|c|c|c|c|c|c|}
\hline
 & $m$ (MeV) & $Q$ & $I$ & $J^P$ & $S$ & Quarks \\
\hline
$p$ & 938.3 & $+1$ & $\frac{1}{2}$ & $\frac{1}{2}^+$ & 0 & $uud$ \\
$n$ & 939.6 & $0$ & $\frac{1}{2}$ & $\frac{1}{2}^+$ & 0 & $udd$ \\
$\Lambda^0$ & 1115.7 & $0$ & 0 & $\frac{1}{2}^+$ & $-1$ & $uds$ \\
$\Sigma^+$ & 1189.4 & $+1$ & 1 & $\frac{1}{2}^+$ & $-1$ & $uus$ \\
$\Sigma^0$ & 1192.6 & $0$ & 1 & $\frac{1}{2}^+$ & $-1$ & $uds$ \\
$\Sigma^-$ & 1197.4 & $-1$ & 1 & $\frac{1}{2}^+$ & $-1$ & $dds$ \\
$\Xi^0$ & 1314.9 & $0$ & $\frac{1}{2}$ & $\frac{1}{2}^+$ & $-2$ & $uss$ \\
$\Xi^-$ & 1321.7 & $-1$ & $\frac{1}{2}$ & $\frac{1}{2}^+$ & $-2$ & $dss$ \\
$\Omega^-$ & 1672.4 & $-1$ & 0 & $\frac{3}{2}^+$ & $-3$ & $sss$ \\
$\Delta^{++}$ & 1232 & $+2$ & $\frac{3}{2}$ & $\frac{3}{2}^+$ & 0 & $uuu$ \\
$\Delta^+$ & 1232 & $+1$ & $\frac{3}{2}$ & $\frac{3}{2}^+$ & 0 & $uud$ \\
$\Delta^0$ & 1232 & $0$ & $\frac{3}{2}$ & $\frac{3}{2}^+$ & 0 & $udd$ \\
$\Delta^-$ & 1232 & $-1$ & $\frac{3}{2}$ & $\frac{3}{2}^+$ & 0 & $ddd$ \\
\hline
\end{supertabular}

\section*{Bosones}
$B = 0$, $L = 0$

\begin{supertabular}{|c|c|c|c|c|c|c|}
\hline
 & $m$ (GeV) & $Q$ & $J^P$ & $C$ & Tipo & Quarks \\
\hline
$\gamma$ & 0 & 0 & $1^-$ & $-1$ & EM & - \\
$g$ & 0 & 0 & $1^-$ & $-1$ & Fuerte & - \\
$Z^0$ & 91.2 & 0 & $1^-$ & - & Débil & - \\
$W^\pm$ & 80.4 & $\pm1$ & $1^-$ & - & Débil & - \\
$H^0$ & 125.1 & 0 & $0^+$ & $+1$ & Escalar & - \\
\hline
\end{supertabular}

\end{document}